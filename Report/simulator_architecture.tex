Our simulator is organized in a main file and several other files that serves as functions for the main one.
In the follows a brief explanation of each part of the script is presented. For the sake of simplicity, a one-satellite simulator is taken into account, the extension to a multi-satellite model is explained later.
The main file is organized in different sections where external functions are called:
	\begin{itemize}
		\item Initialization of all the fixed parameters used in the simulation;
		\item computation of the trajectory of a satellite in the chosen orbit in terms of Orbital Coordinates system;
		\item \gls{eci} and \gls{lla} coordinates are computed;
		\item plot of a 3D animation in which the satellite and its trajectory are shown;
		\item plot of the Ground Track of the satellite;
		\item estimation of azimuth and elevation of the satellite viewed from the \gls{gs} position;
		\item link budget estimation.
	\end{itemize}

In case of more than one satellite, each one has to cover the same area of the Earth but in different moments, and since the Earth rotates, a simple delay in the same orbital plane is not enough.

The solution we found for this problem is a delay in the time (with the same trajectory) and a different \gls{raan} for each satellite.
The \gls{raan}'s offset angle of each satellite is given by the orbital period ($T$) of the orbit.
\textit{Tundra} is a Geosynchronous Orbit, so its orbital period is the same of the Earth, and the following formulas have been used:

\begin{equation}
	d^{time}(s) = \frac{T}{n} \qquad d^{raan}(deg) = \frac{360}{n}
\end{equation}

Where $T$ is the orbital period and $n$ is the number of satellites simulated.

In case of a \textit{Molniya} orbit, the orbital period is an half the revolution period of the Earth, so the \gls{raan} has to be an half of the one calculated for \textit{Tundra}. In formulas:

\begin{equation}
	d^{time}(s) = \frac{T}{n} \qquad d^{raan}(deg) = \frac{360}{2n} = \frac{180}{n}
\end{equation}

With a multi-satellite system, the \gls{gs} has to communicate each time with the best satellite, i.e. the satellite with the higher elevation. Based on this assumption, the best satellite in each instant is calculated in the script and the actual elevation and azimuth of the \gls{gs}'s antenna is plotted.

Finally, the Overall Link Budget for a \gls{gs} in each instant is the one calculated between the \gls{gs} itself and the best satellite in that moment.
