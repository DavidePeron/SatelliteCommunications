\documentclass[11pt,a4paper,titlepage]{article}
\usepackage[a4paper]{geometry}
\usepackage[utf8]{inputenc}
\usepackage[english]{babel}
\usepackage{lipsum}
\usepackage{eurosym}
\usepackage{rotating}

\usepackage{amsmath, amssymb, amsfonts, amsthm, mathtools}
% mathtools for: Aboxed (put box on last equation in align envirenment)
\usepackage{microtype} %improves the spacing between words and letters

\usepackage{lipsum}
\usepackage{threeparttable}
\usepackage{tabularx}
\usepackage{multirow}
\usepackage{booktabs}
\newcommand{\tabitem}{~~\llap{\textbullet}~~}
\usepackage{graphicx}
\graphicspath{ {./figures/} {./eps/}}
\usepackage{epsfig}
\usepackage{epstopdf}
\usepackage{verbatim}
\usepackage{textcomp}
\usepackage{tikz}
\usetikzlibrary{shapes,arrows}

%%%%%%%%%%%%%%%%%%%%%%%%%%%%%%%%%%%%%%%%%%%%%%%%%%
%% COLOR DEFINITIONS
%%%%%%%%%%%%%%%%%%%%%%%%%%%%%%%%%%%%%%%%%%%%%%%%%%
 % Enabling mixing colors and color's call by 'svgnames'
%%%%%%%%%%%%%%%%%%%%%%%%%%%%%%%%%%%%%%%%%%%%%%%%%%
\definecolor{MyColor1}{rgb}{0.2,0.4,0.6} %mix personal color
\newcommand{\textb}{\color{Black} \usefont{OT1}{lmss}{m}{n}}
\newcommand{\blue}{\color{MyColor1} \usefont{OT1}{lmss}{m}{n}}
\newcommand{\blueb}{\color{MyColor1} \usefont{OT1}{lmss}{b}{n}}
\newcommand{\red}{\color{LightCoral} \usefont{OT1}{lmss}{m}{n}}
\newcommand{\green}{\color{Turquoise} \usefont{OT1}{lmss}{m}{n}}
%%%%%%%%%%%%%%%%%%%%%%%%%%%%%%%%%%%%%%%%%%%%%%%%%%


%%%%%%%%%%%%%%%%%%%%%%%%%%%%%%%%%%%%%%%%%%%%%%%%%%
%% FONTS AND COLORS
%%%%%%%%%%%%%%%%%%%%%%%%%%%%%%%%%%%%%%%%%%%%%%%%%%
%    SECTIONS
%%%%%%%%%%%%%%%%%%%%%%%%%%%%%%%%%%%%%%%%%%%%%%%%%%
\usepackage{titlesec}
\usepackage{sectsty}
%%%%%%%%%%%%%%%%%%%%%%%%
%set section/subsections HEADINGS font and color
\sectionfont{\color{MyColor1}}  % sets colour of sections
\subsectionfont{\color{MyColor1}}  % sets colour of sections

%set section enumerator to arabic number (see footnotes markings alternatives)
\renewcommand\thesection{\arabic{section}.} %define sections numbering
\renewcommand\thesubsection{\thesection\arabic{subsection}} %subsec.num.

%define new section style
\newcommand{\mysection}{
\titleformat{\section} [runin] {\usefont{OT1}{lmss}{b}{n}\color{MyColor1}}
{\thesection} {3pt} {} }

%%%%%%%%%%%%%%%%%%%%%%%%%%%%%%%%%%%%%%%%%%%%%%%%%%
%		CAPTIONS
%%%%%%%%%%%%%%%%%%%%%%%%%%%%%%%%%%%%%%%%%%%%%%%%%%
\usepackage{caption}
\usepackage{subcaption}
%%%%%%%%%%%%%%%%%%%%%%%%
\captionsetup[figure]{labelfont={color=MyColor1}}

%%%%%%%%%%%%%%%%%%%%%%%%%%%%%%%%%%%%%%%%%%%%%%%%%%
%		!!!EQUATION (ARRAY) --> USING ALIGN INSTEAD
%%%%%%%%%%%%%%%%%%%%%%%%%%%%%%%%%%%%%%%%%%%%%%%%%%
%using amsmath package to redefine eq. numeration (1.1, 1.2, ...)
%%%%%%%%%%%%%%%%%%%%%%%%
\renewcommand{\theequation}{\thesection\arabic{equation}}

%set box background to grey in align environment
\usepackage{etoolbox}% http://ctan.org/pkg/etoolbox
\makeatletter
\patchcmd{\@Aboxed}{\boxed{#1#2}}{\colorbox{black!15}{$#1#2$}}{}{}%
\patchcmd{\@boxed}{\boxed{#1#2}}{\colorbox{black!15}{$#1#2$}}{}{}%
\makeatother
%%%%%%%%%%%%%%%%%%%%%%%%%%%%%%%%%%%%%%%%%%%%%%%%%%

\newcommand{\DP}[1]{\textcolor{blue}{\textbf{(DP says: #1)}}}
\newcommand{\cri}[1]{\textcolor{green}{\textbf{(Cri says: #1)}}}

\makeatletter
\let\reftagform@=\tagform@
\def\tagform@#1{\maketag@@@{(\ignorespaces\textcolor{red}{#1}\unskip\@@italiccorr)}}
\renewcommand{\eqref}[1]{\textup{\reftagform@{\ref{#1}}}}
\makeatother
\usepackage[hidelinks]{hyperref}

%% LISTS CONFIGURATION %%
\usepackage{enumitem}
\setlist[enumerate,1]{start=0}
\renewcommand{\labelenumii}{\theenumii}
\renewcommand{\theenumii}{\theenumi.\arabic{enumii}.}

\usepackage[acronym]{glossaries}
\newacronym[plural=GEO,longplural={Geostationary Earth Orbits}]{geo}{GEO}{Geostationary Earth Orbit}
\newacronym[plural=LEO,longplural={Low Earth Orbits}]{leo}{LEO}{Low Earth Orbit}
\newacronym[plural=MEO,longplural={Medium Earth Orbits}]{meo}{MEO}{Medium Earth Orbit}
\newacronym[plural=HEO,longplural={High Elliptical Orbits}]{heo}{HEO}{High Elliptical Orbit}
\newacronym[plural=GS,longplural={Ground Stations}]{gs}{GS}{Ground Station}

%%%%%%%%%%%%%%%%%%%%%%%%%%%%%%%%%%%%%%%%%%%%%%%%%%
%% PREPARE TITLE
%%%%%%%%%%%%%%%%%%%%%%%%%%%%%%%%%%%%%%%%%%%%%%%%%%
\title{\blue Satellite Communications \\
\blueb Satellite system to provide communication services to polar regions in Europe and Russia}
\author{Ana Reviejo Jiménez \\ Marta Munilla Díez\\ Oscar Pla Terrada\\ Davide Peron\\ Cristina Gava\\ Javier Garcia Camin}
\date{\today}
%%%%%%%%%%%%%%%%%%%%%%%%%%%%%%%%%%%%%%%%%%%%%%%%%%

\begin{document}
\maketitle

\tableofcontents
\clearpage

\section{Problem Description} \label{sec:problem_description}
	\begin{figure}
	\centering
	\begin{minipage}{0.45\textwidth}
		\includegraphics[width=\textwidth]{figures/System_topology_noBG.png}
		\caption{Scheme of the topology of the system.}
		\label{fig:topology}
	\end{minipage}\hspace{0.5cm}
	\begin{minipage}{0.45\textwidth}
		\includegraphics[width=\textwidth]{figures/System_topology_noBG.png}
		\caption{Typical communication path between an user A and an user B.}
		\label{fig:communication}
	\end{minipage}
\end{figure}

This project results from the necessity of having a good broadband coverage of polar
areas and the land areas of Northern Europe and Russia: this means the coverage of
latitudes over the 60 deg.

The subjects interested in this kind of communication are
mostly industries involved in economic sector: they need a reliable communication system able to provide a service of 50 Mbps in download and 5 Mbps in upload.


The aim is to project a system able to provide a continuous, reliable and feasible communication service, maximizing the number of users allowed to access it over $60^\circ$
latitudes and minimizing the costs. To do that, services in narrowband communication using LEO satellites are not useful, since the broadband communication required is not feasible with this technology.

A simple representation of the system to be built is shown in \autoref{fig:topology} and a communication between two users is in \autoref{fig:communication}.

Typically, if a user A has to communicate with user B, it sends his packets to the satellite, with the recipient address in the header.
The satellite receives the packets and forwards them to the Ground Station that sends them to the proper application (Skype, Hangout, ...).
These packets are sent from the application to the Ground Station, that forwards them, through the satellite, to the recipient B.


\section{Simulator and Orbits} \label{sec:orbit}
	To guarantee the service required in \autoref{sec:problem_description}, different orbits have been taken in account.
	The most used orbit to ensure a stable and reliable satellite communication is Geostationary.
	\autoref{fig:GEOCoverage} has been taken from the Inmarsat's Website, and it shows as a \gls{geo} satellite can not reach the latitudes over $75^\circ$. For this reason a \gls{geo} does not fit our purpose.

	\glspl{leo} has been discarded since the time of visibility for a single satellite is very low, so an high number of satellites and an accurate tracking system are required to ensure a continuous service.

	\glspl{meo} suffer the same problems of \glspl{leo} ones, with the addition of the proximity to the Van Allen Belt where signal degradation increases significantly.

	The most suitable solution for our problem is a \gls{heo}.

	\begin{figure}
		\centering
		\includegraphics[width=0.7\textwidth]{figures/GEOCoverage.jpeg}
		\caption{Approximate coverage of GEO Satellites.}
		\label{fig:GEOCoverage}
	\end{figure}
	\subsection{Simulator Architecture}
		\lipsum[2]
	\subsection{Orbit selection}
		To guarantee the service required in \autoref{sec:problem_description}, different orbits have been taken in account.
The most used orbit to ensure a stable and reliable satellite communication is Geostationary.
\autoref{fig:GEOCoverage} has been taken from the Inmarsat's Website, and shown as a GEO satellite can not reach the latitudes over $75^\circ$. 

\begin{figure}
	\centering
	\begin{minipage}{0.45\textwidth}
		\includegraphics[width=\textwidth]{figures/GEOCoverage.jpeg}
		\caption{Approximate coverage of GEO Satellites.}
		\label{fig:GEOCoverage}
	\end{minipage}\hspace{0.5cm}
	\begin{minipage}{0.45\textwidth}
		\includegraphics[width=\textwidth]{figures/System_topology_noBG.png}
		\caption{Typical communication path between an user A and an user B.}
	\end{minipage}
\end{figure}


\section{Payload and Space Segment}
		For the space segment a payload a considerations on the transponders have been made based on the requirements that the mission has to satisfy: to be precise, the problem description requires a broadcast communication that guarantees a capacity of 5 Mbit in uplink and of 50 Mbit in downlink; moreover the communication has to allow the internet connection and video and voice call service.
	\subsection{Communication Module}
		The first thing to do was to select the transponder size, which we fixed at 72 MHz. After that we decided the number of carriers in for the forward and the return link and the amplitude of the guard-band between the carriers. The resulting values are listed in \autoref{tab:commModule} and \autoref{fig:transp} shows a schematic representation of a transponder.

		\begin{table}
		\centering
		\begin{tabular}{lr}
		\toprule
		Feature & Value\\
		\midrule
		Transponder size & 72 MHz\\
		N carriers in forward link & 2\\
		N carriers in return link & 6\\
		Amplitude carriers in fw & 27.805 MHz\\
		Amplitude carriers in rt & 2,732 MHz\\
		Tot. fw link bandiwdth & 55.61 MHz\\
		Tot. rt link bandwidth & 16.39 MHz\\
		Guard-band & 3.6 MHz\\
		Tot. bandwidth used & 450 MHz\\
		\bottomrule 
		\end{tabular}
		\caption{Values for the communication module}
		\label{tab:commModule}
		\end{table}
		
		\begin{figure}
		\centering
		\includegraphics[width = .8\textwidth]{Transponder.png}
		\caption{Representation of a transponder}
		\label{fig:transp}
		\end{figure}

		Through these values the total number of transponder we have on the satellite is 12, 6 with horizontal polarization and 6 with the vertical one.

	\subsection{Frequency Plan}
After these considerations the structure of the Frequency Plan is automatically elaborated and here it is represented in \autoref{fig:freqPlan}

		\begin{figure}
		\centering
		\includegraphics[width = .65\textwidth]{Frequencies.png}
		\caption{Frequency plan for the communication module}
		\label{fig:freqPlan}
		\end{figure}

	\subsection{Payload}
		%% Definition of blocks:
\tikzset{%
  block/.style    = {draw, thick, rectangle, minimum height = 3em,
    minimum width = 3em},
	rect/.style    = {draw, thick, rectangle, minimum height = 3em,
	    minimum width = 1.5em},
	mux/.style    = {draw, thick, rectangle, minimum height = 7em,
			minimum width = 2.5em, align=left},
	triang/.style    = {draw, thick, isosceles triangle, minimum height = 3em, minimum width = 1.5em, align=left},
  mult/.style      = {draw, circle, node distance = 2.7cm},
  ghost/.style    = {coordinate}, % Input
  output/.style   = {coordinate} % Output
}
% Defining string as labels of certain blocks.
\newcommand{\mult}{\Large$\times$}
\newcommand{\inte}{$\displaystyle \int$}
\newcommand{\derv}{\huge$\frac{d}{dt}$}

\begin{tikzpicture}[auto, thick, node distance=2cm, >=triangle 45]
\draw
	% Drawing the blocks of first filter :
	node at (0,0)[right=-3mm, , label={below:(0)}]{\Large \textopenbullet}
	node [ghost, name=input1] {}
	% node [sum, right of=input1] (suma1) {\suma}
	node [rect, right of=input1, label={below:(1)}] (pol_sep) {}
  node [triang, right of=pol_sep, label={below:(2)}] (lna) {LNA}
  node [mult, right of=lna, label={D/C}, label={below:(3)}] (dlc) {\mult}
	node [triang, right of=dlc, label={below:(4)}] (ifa) {IF \\ amp}
	node [mult, right of=ifa, label={U/C}, label={below:(5)}] (ulc) {\mult}
	node [triang, right of=ulc, label={below:(6)}] (hpa) {HPA}
	node [triang, right of=hpa, label={below:(7)}] (bpf) {}
	node [mux, right of=bpf] (imux) {I \\M \\U \\ X}
	node at (19,1.5)[right=-3mm, name = ch1, label={left:$ch_1$}]{\Large \textopenbullet}
	node at (19,1)[right=-3mm, name = ch2]{}
	node at (19,0.5)[right=-3mm, name = ch3]{}
	node at (19,0)[right=-3mm, name = ch4]{}
	node at (19,-1.5)[right=-3mm, name = chN, , label={left:$ch_N$}]{}
	node [triang, right of=ch1, label={below:(8)}] (ca) {}
	node [block, right of=ca, label={below:(9)}] (alc) {ALC}
	node [triang, right of=alc, label={below:(10)}] (outamp) {}
	node [mux, right of=imux, node distance = 10cm] (omux) {O \\M \\U \\ X}
	node [triang, right of=omux, label={below:(11)}] (bpf2) {}
	node at (32,0)[right=-3mm, name = outantenna, label={below:(12)}]{\Large \textopenbullet};
    % Joining blocks.
    % Commands \draw with options like [->] must be written individually
	\draw[-](input1) -- node {}(pol_sep);
	\draw[-](pol_sep) -- node {} (lna);
	\draw[-](lna) -- node {$f_D$} (dlc);
	\draw[-](dlc) -- node {$f_{IF}$} (ifa);
	\draw[-](ifa) -- node {$f_{IF}$} (ulc);
	\draw[-](ulc) -- node {$f_u$} (hpa);
	% \draw[-](rx) -- node {} (bpf);
	\draw[-](hpa) -- node {} (bpf);
	\draw[-](bpf) -- node {} (imux);
	\draw[-](imux) -- node {} (ch1);
	\draw[-](imux) -- node {} (ch2);
	\draw[dashed](imux) -- node {} (ch3);
	\draw[dashed](imux) -- node {} (ch4);
	\draw[-](imux) -- node {} (chN);
	\draw[-](ch1) -- node {} (ca);
	\draw[-](ca) -- node {} (alc);
	\draw[-](alc) -- node {} (outamp);
	\draw[-](outamp) -- node {} (omux);
	\draw[-](omux) -- node {} (bpf2);
	\draw[-](bpf2) -- node {} (outantenna);

% 	% Boxing and labelling
	\draw [color=gray, dashed, label={Receiver Block}](1,-1.5) rectangle (14.5,1.5);
	\node at (6.5,1.5) [above=5mm, right=0mm] {\textsc{Receiver Block}};

	\draw [color=gray, dashed, label={Receiver Block}](14.8,2.5) rectangle (31,-2.5);
	\node at (21.5,2.5) [above=5mm, right=0mm] {\textsc{Repeater Block}};
	\draw [color=gray,thick](-0.5,-9) rectangle (12.5,-5);
	\node at (-0.5,-9) [below=5mm, right=0mm] {\textsc{second-order noise shaper}};
\end{tikzpicture}
		The electronic part of the payload is composed by the two main sections of the \textbf{receiver block} and \textbf{repeater block}: in the receiver part, the signal is received, separated in polarization, filtered and amplified so as to be ready for the repeater part, in which it is channelized and further amplified. \autoref{fig.payload} shows the global representation of the payload.

		\begin{sidewaysfigure}
		\centering
		\includegraphics[width = \textwidth]{Payload.png}
		\caption{Payload representaiton}
		\label{fig:payload}
		\end{sidewaysfigure}

		\subsubsection{Receiver Block}
			\begin{figure}
			\centering
			\includegraphics[width = .65\textwidth]{Frequencies.png}
			\caption{Frequency plan for the communication module}
			\label{fig:freqPlan}
			\end{figure}
		\subsubsection{Repeater Block}
			\lipsum[1]
	\subsection{Power Budget}
		\subsubsection{Required Power}
			\lipsum[1]
		\subsubsection{Solar Panels specifications}
			\lipsum[1]
	\subsection{Weight Estimation}
		\lipsum[1]

\section{Ground Segment}
	The ground segment is composed by a single \gls{gs} and all the mobile users moving around on the northern lands and 		seas of Russia and Europe. The reason for just one \gls{gs} is that, since we do not deploy the multibeam option, there is no 		necessity of having more than one \gls{gs}, because one is enough to correctly collect all the traffic coming from the satellite 		and toward it.
	\subsection{Ground Station coordinates}
		The coordinates of the \gls{gs} are: \cri{Cambia valore quando Davide ha trovato un buon posto per la GS}
		\begin{align}
		lat &= 61.267865\\ 
		long &= -96.608223\\
		\end{align}
		These coordinates have been chosen based on the elevation values: the coordinates listed above represent a point in 				Russia in the surrounding area of the subsatellite point of the apogee. In this way, the values of elevation are always 				substantially high, guaranteeing always a good visibility. Moreover, another factor is the rain attenuation, which is not 			much high in this region, as we can conclude from the respective value of $R$:  \cri{Cambia valore quando Davide ha trovato un buon posto per la GS}
		\begin{equation}
		R = 22.2016 \quad mm/h
		\end{equation}
	\subsection{Ground Station requirements}
		The requirements for the \gls{gs} are substantially the following:
		\begin{itemize}
			\item the Internet connection;
			\item the antenna model and specifications.
		\end{itemize}

		The Internet connection is the most important (if not the only one) reason for the existence of the \gls{gs}, since one of 			the problem requirements is indeed the possibility of video calls and other internet services.

		For the antenna model we chose a reflector antenna with a single circular beam; the antenna parameters are 					listed in \autoref{tab:antennaParam}
		\begin{table}
			\centering
			\begin{tabular}{cc}
			\toprule
			Parameter & Value\\
			\midrule
			Frequency Band & Ku Band\\
			Dish diameter D & 11 m\\
			Efficiency & 0.6\\
			IBO & -0.5 \cri{può essere considerato un parametro d'antenna?}\\
			\bottomrule
			\end{tabular}
			\caption{\gls{gs} antenna specifications}
			\label{tab:antennaParam}
		\end{table}

		%Another aspect to consider is the fact that the variations in the azimuth values are very steep in a specific moment of the satellite revolution, and this brings as side effect the temporary suspension of the service for the time needed by the \gls{gs} to point properly the correct satellite in orbit (about 2 minutes).
\cri{la metto sta frase qui? a suo tempo il prof mi pare avesse detto che andava specificato, però a leggerla mi rattristo a pensare che offriamo un servizio che per 2 minuti al giorno sistematicamente non funziona}
	\subsection{User requirements}
The user requirements are substantially the model of the antenna and the dimension of the dish: the model is identical to the ones used for the satellite and the \gls{gs}, so a reflector antenna with a single beam; the dish diameter is smaller in this case, for a matter of space and feasibility, and is of 1 m \cri{verifica valore}. There is no need, in this case, of specifying the position of the users since by definition they are mobile users.

\section{Link Budget}
	\subsection{Parameters setting and estimation}
		\subsubsection{Antenna Parameters}
		\subsubsection{Effective Isotropic Radiated Power(EIRP)}
		\subsubsection{Losses}
	\subsection{Uplink}
	\subsection{Downlink}
	\subsection{Overall Link Budget}
\lipsum[1]

\section{Cost Estimation}
	\subsection{Spacecraft cost}
		The spacecraft cost can be estimated depending on several parameters and criteria, such as the type of mission, the 				subsystem considered and the unit over which calculate the cost. In our specific case we concentrated on the cost analysis 		for a communication-type satellite and review it for every subsystem of the spacecraft and its launch procedure.\\
		
		The subsystems analyzed  are the following:
		\begin{itemize}
			\item Attitude determination and Control subsystem (ADCS)
			\item Communication subsystem
			\item Electrical power subsystem (EPS)
			\item Integration assembly and test (IA\&T)
			\item Passive sensor
			\item Propulsion
			\item System engineering
			\item Structure
			\item Thermal control
			\item Telemetry tracking and command (TT\&C)
		\end{itemize}
		In particular, \autoref{fig:torta} shows the cost percentage that each system represents: from it we can see that the 				System engineering is the most important item, followed by the EPS and the IA\&T subsystems. Moreover, 						\autoref{fig:distribution} lists the different sections, depending on the type of mission the satellite is intended to 					accomplish, with their standard deviations; tables \ref{fig:mission} and \ref{fig:mission_pound}, instead, show the total 			cost depending on the mission type and the total cost per pound.
		
		\begin{figure}
			\centering
			\includegraphics[width = .7\textwidth]{Torta.png}
			\caption{Communication spacecraft cost composition}
			\label{fig:torta}
		\end{figure}
		
		\begin{figure}
			\centering
			\includegraphics[width = 1\textwidth]{Standard_dev.png}
			\caption{Communication spacecraft cost composition: averages and standard deviations}
			\label{fig:distribution}
		\end{figure}
		
		\begin{figure}
			\centering
			\begin{minipage}{1\textwidth}
			\centering
			\includegraphics[width = .95\textwidth]{mission.png}
			\caption{Total spacecraft cost}
			\label{fig:mission}
			\end{minipage}
			\hspace{20mm}
			\begin{minipage}{.95\textwidth}
			\centering
			\includegraphics[width = .95\textwidth]{mission_pound.png}
			\caption{Total spacecraft cost per pound}
			\label{fig:mission_pound}
			\end{minipage}
		\end{figure}
		
		Regarding the cost per subsystem, \autoref{tab:systems1} and \autoref{tab:systems2} show the different cost each 				subsystem is intended to have:
		
		\begin{table}
			\centering
			\begin{tabular}{ccc}
			\toprule
			Subsystem & Mean Cost (k\euro) & Standard deviation\\
			\midrule
			IA\&T     & 8311,49   & 8719,94\\
			EPS        & 8441,34   & 5681,80\\
			Structure & 4111,49   & 2955,92\\
			SEPM      & 12167,05 & 7825,63\\
			Thermal  & 903,45    & 562,3\\
			TT\&C    & 4423,24   & 2942,24\\ 
			\bottomrule
			\end{tabular}
			\caption{List of the costs per subsystem}
			\label{tab:systems1}
		\end{table}
		
		\begin{table}
			\centering
			\begin{tabular}{ccc}
			\toprule
			Subsystem & Mean Cost/unit (k\euro/kg or ch) & Standard deviation\\
			\midrule
			ADCS                                         & 94,70     & 8719,94\\
			Communication ($1 < ch < 10$)   & 3923,19 & 1443,98\\
			Communication ($10 < ch < 25$) & 1534,45 & 558,37\\
			Communication ($25 < ch$)         & 708,40   & 197,35\\
			EPS                                            & 24,7      & 7,27\\
			Propulsion                                  & 54,68     & 14,32\\
			Structure                                    & 15,94     & 4,37\\
			\bottomrule
			\end{tabular}
			\caption{List of the costs per subsystem per pound/channel}
			\label{tab:systems2}
		\end{table}
		
		Through this data we can make a raw hypothesis on the average total cost of the spacecraft with a summary estimation 			of its mass:
		
		\begin{table}
			\centering
			\begin{tabular}{ccc}
			\toprule
			\multicolumn{3}{c}{Communication spacecraft}\\
			\midrule
			IA\&T       & 8311,49 \euro       & $+$\\
			EPS          & 24,7 \euro/Kg        & $\times NCHILI +$\\
			Structure   & 15,94  \euro/Kg     & $\times NCHILI +$\\
			SEPM        & 12167,05 \euro     & $+$\\
			Thermal    & 903,45 \euro        & $+$\\
			TT\&C       & 4423,24 \euro      & $+$\\ 
			ADCS        & 94,70 \euro/Kg     & $\times NCHILI +$\\
			Propulsion & 54,68   \euro/Kg   & $\times NCHILI +$\\
			Communication ($10 < ch < 25$) & 1534,45 \euro/ch & $\times 12 ch =$\\
			\bottomrule
			Total cost:& & TOT\\
			\end{tabular}
			\caption{List of the costs per subsystem per pound/channel}
			\label{tab:cost}
		\end{table}
	\subsection{Launch cost}
For the launch cost we based our considerations on the prices listed by the $SpaceX$ company. \autoref{fig:spacex} shows the prices for different types of launches, depending on the mass of the spacecrafts and the orbits they should reach.

Through the considerations we have made in the previous sections we can state that around $180$ Millions of dollars ($151.793.055$ \euro \cri{verifica il prezzo}) are needed for the launch: in fact each spacecraft has a total mass of about \cri{mettere massa} and the Molniya orbit is a HEO orbit; moreover, since the raans of the two orbital planes are separated of 180 $\deg$ it is necessary to use two separate launchers, one for each spacecraft.\\

Through this analysis the total cost for the project is:
\begin{equation}
Cost_{Total} = Cost_{Launch} + Cost_{Spacecraft} = \cri{Mettere costo finale} \text{\euro}
\end{equation}

\begin{figure}
\centering
\includegraphics[width = .9\textwidth]{Spacex.png}
\caption{$SpaceX$ price list}
\label{fig:spacex}
\end{figure}

\section{Final considerations and conclusions}
	\lipsum[1]























\end{document}
