% Definition of blocks:
\tikzset{%
  block/.style    = {draw, thick, rectangle, minimum height = 3em,
    minimum width = 3em},
	rect/.style    = {draw, thick, rectangle, minimum height = 3em,
	    minimum width = 1.5em},
	mux/.style    = {draw, thick, rectangle, minimum height = 7em,
			minimum width = 2.5em, align=left},
	triang/.style    = {draw, thick, isosceles triangle, minimum height = 3em, minimum width = 1.5em, align=left},
  mult/.style      = {draw, circle, node distance = 2.7cm},
  ghost/.style    = {coordinate}, % Input
  output/.style   = {coordinate} % Output
}
% Defining string as labels of certain blocks.
\newcommand{\mult}{\Large$\times$}
\newcommand{\inte}{$\displaystyle \int$}
\newcommand{\derv}{\huge$\frac{d}{dt}$}

\begin{tikzpicture}[auto, thick, node distance=2cm, >=triangle 45]
\draw
	% Drawing the blocks of first filter :
	node at (0,0)[right=-3mm, , label={below:(0)}]{\Large \textopenbullet}
	node [ghost, name=input1] {}
	% node [sum, right of=input1] (suma1) {\suma}
	node [rect, right of=input1, label={below:(1)}] (pol_sep) {}
  node [triang, right of=pol_sep, label={below:(2)}] (lna) {LNA}
  node [mult, right of=lna, label={D/C}, label={below:(3)}] (dlc) {\mult}
	node [triang, right of=dlc, label={below:(4)}] (ifa) {IF \\ amp}
	node [mult, right of=ifa, label={U/C}, label={below:(5)}] (ulc) {\mult}
	node [triang, right of=ulc, label={below:(6)}] (hpa) {HPA}
	node [triang, right of=hpa, label={below:(7)}] (bpf) {}
	node [mux, right of=bpf] (imux) {I \\M \\U \\ X}
	node at (19,1.5)[right=-3mm, name = ch1, label={left:$ch_1$}]{\Large \textopenbullet}
	node at (19,1)[right=-3mm, name = ch2]{}
	node at (19,0.5)[right=-3mm, name = ch3]{}
	node at (19,0)[right=-3mm, name = ch4]{}
	node at (19,-1.5)[right=-3mm, name = chN, , label={left:$ch_N$}]{}
	node [triang, right of=ch1, label={below:(8)}] (ca) {}
	node [block, right of=ca, label={below:(9)}] (alc) {ALC}
	node [triang, right of=alc, label={below:(10)}] (outamp) {}
	node [mux, right of=imux, node distance = 10cm] (omux) {O \\M \\U \\ X}
	node [triang, right of=omux, label={below:(11)}] (bpf2) {}
	node at (32,0)[right=-3mm, name = outantenna, label={below:(12)}]{\Large \textopenbullet};
    % Joining blocks.
    % Commands \draw with options like [->] must be written individually
	\draw[-](input1) -- node {}(pol_sep);
	\draw[-](pol_sep) -- node {} (lna);
	\draw[-](lna) -- node {$f_D$} (dlc);
	\draw[-](dlc) -- node {$f_{IF}$} (ifa);
	\draw[-](ifa) -- node {$f_{IF}$} (ulc);
	\draw[-](ulc) -- node {$f_u$} (hpa);
	% \draw[-](rx) -- node {} (bpf);
	\draw[-](hpa) -- node {} (bpf);
	\draw[-](bpf) -- node {} (imux);
	\draw[-](imux) -- node {} (ch1);
	\draw[-](imux) -- node {} (ch2);
	\draw[dashed](imux) -- node {} (ch3);
	\draw[dashed](imux) -- node {} (ch4);
	\draw[-](imux) -- node {} (chN);
	\draw[-](ch1) -- node {} (ca);
	\draw[-](ca) -- node {} (alc);
	\draw[-](alc) -- node {} (outamp);
	\draw[-](outamp) -- node {} (omux);
	\draw[-](omux) -- node {} (bpf2);
	\draw[-](bpf2) -- node {} (outantenna);

% 	% Boxing and labelling
	\draw [color=gray, dashed, label={Receiver Block}](1,-1.5) rectangle (14.5,1.5);
	\node at (6.5,1.5) [above=5mm, right=0mm] {\textsc{Receiver Block}};

	\draw [color=gray, dashed, label={Receiver Block}](14.8,2.5) rectangle (31,-2.5);
	\node at (21.5,2.5) [above=5mm, right=0mm] {\textsc{Repeater Block}};
	\draw [color=gray,thick](-0.5,-9) rectangle (12.5,-5);
	\node at (-0.5,-9) [below=5mm, right=0mm] {\textsc{second-order noise shaper}};
\end{tikzpicture}
