The ground segment is composed by a single \gls{gs} and all the mobile users moving around on the northern lands and     seas of Russia and Europe. The reason for just one \gls{gs} is that, since we do not deploy the multibeam option, there is no necessity of having more than one \gls{gs}, because one is enough to correctly collect all the traffic coming from the satellite and toward it.
\subsection{Ground Station coordinates}
	The coordinates of the \gls{gs} are: \cri{Cambia valore quando Davide ha trovato un buon posto per la GS}
	\begin{align}
	lat &= 61.267865\\
	long &= -96.608223\\
	\end{align}
	These coordinates have been chosen based on the elevation values: the coordinates listed above represent a point in 				Russia in the surrounding area of the subsatellite point of the apogee. In this way, the values of elevation are always 				substantially high, guaranteeing always a good visibility. Moreover, another factor is the rain attenuation, which is not 			much high in this region, as we can conclude from the respective value of $R$:  \cri{Cambia valore quando Davide ha trovato un buon posto per la GS}
	\begin{equation}
	R = 22.2016 \quad mm/h
	\end{equation}
\subsection{Ground Station requirements}
	The requirements for the \gls{gs} are substantially the following:
	\begin{itemize}
		\item the Internet connection;
		\item the antenna model and specifications.
	\end{itemize}

	The Internet connection is the most important (if not the only one) reason for the existence of the \gls{gs}, since one of 			the problem requirements is indeed the possibility of video calls and other internet services.

	For the antenna model we chose a reflector antenna with a single circular beam; the antenna parameters are 					listed in \autoref{tab:antennaParam}
	\begin{table}
		\centering
		\begin{tabular}{lr}
		\toprule
		Parameter & Value\\
		\midrule
		Frequency Band & Ku Band\\
		Dish diameter D & 6 m\\
		Efficiency & 0.6\\
		IBO & -0.5 \cri{può essere considerato un parametro d'antenna?}\\
		\bottomrule
		\end{tabular}
		\caption{\gls{gs} antenna specifications}
		\label{tab:antennaParam}
	\end{table}

	%Another aspect to consider is the fact that the variations in the azimuth values are very steep in a specific moment of the satellite revolution, and this brings as side effect the temporary suspension of the service for the time needed by the \gls{gs} to point properly the correct satellite in orbit (about 2 minutes).
\cri{la metto sta frase qui? a suo tempo il prof mi pare avesse detto che andava specificato, però a leggerla mi rattristo a pensare che offriamo un servizio che per 2 minuti al giorno sistematicamente non funziona}
\subsection{User requirements}
The user requirements are substantially the model of the antenna and the dimension of the dish: the model is identical to the ones used for the satellite and the \gls{gs}, so a reflector antenna with a single beam; the dish diameter is smaller in this case, for a matter of space and feasibility, and is of 1 m \cri{verifica valore}. There is no need, in this case, of specifying the position of the users since by definition they are mobile users.
